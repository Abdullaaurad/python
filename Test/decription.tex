\documentclass{article}
\usepackage{listings}
\usepackage{xcolor}
\usepackage[margin=1cm]{geometry}

% Define colors for syntax highlighting
\definecolor{codegreen}{rgb}{0,0.6,0}
\definecolor{codegray}{rgb}{0.5,0.5,0.5}
\definecolor{codepurple}{rgb}{0.58,0,0.82}
\definecolor{backcolour}{rgb}{0.95,0.95,0.92}

% Define Python language for listings package
\lstdefinestyle{python}{
    language=Python,
    backgroundcolor=\color{backcolour},
    commentstyle=\color{codegreen},
    keywordstyle=\color{blue},
    numberstyle=\tiny\color{codegray},
    stringstyle=\color{codepurple},
    basicstyle=\ttfamily\small,
    breaklines=true,
    breakatwhitespace=true,
    numbers=left,
    showstringspaces=false,
    captionpos=b,
}

\begin{document}

\title{Writing Test Cases in Python}
\author{Your Name}
\date{\today}
\maketitle

\section{Introduction}

This document demonstrates how to write and run test cases in Python using various techniques.

\subsection{Naming Conventions}

Test files should follow the naming convention \texttt{test\_<module\_name>.py}. For example, if your module is named \texttt{calculator.py}, your test file should be named \texttt{test\_calculator.py}.

\subsection{Importing Modules}

In your test file (\texttt{test\_<module\_name>.py}), import the necessary modules and functions/classes from your main Python file (\texttt{<module\_name>.py}) that you want to test. For example:

\begin{lstlisting}[style=python]
from calculator import add_numbers
\end{lstlisting}

\subsection{Writing Test Cases}

Use a testing framework like \texttt{unittest} or \texttt{pytest} to write your test cases. Here's an example using \texttt{unittest}:

\begin{lstlisting}[style=python]
import unittest
from calculator import add_numbers

class TestCalculator(unittest.TestCase):
    def test_add_numbers(self):
        result = add_numbers(3, 5)
        self.assertEqual(result, 8)  # Assert that the result is equal to the expected value
\end{lstlisting}

\subsection{Running the Tests}

After writing your test cases, run them using a test runner. For \texttt{pytest}, you would run the tests with:

\begin{lstlisting}[language=bash]
pytest test_calculator.py
\end{lstlisting}

To run a specific function in the test file
\begin{lstlisting}[language=bash]
pytest test_Math.py::test_Add
\end{lstlisting}

\begin{lstlisting}[language=bash]
pytest -k "add" test_Math.py               #runs all functions that has 'Add' init
pytest -k "add or string" test_Math.py     # runs 'Add' or 'string' init
\end{lstlisting}

to run all the test cases and return a explanation if a case fails
\begin{lstlisting}[language=bash]
pytest -v -x
\end{lstlisting}

\newpage
\section{Complex Test Example}

\begin{lstlisting}[style=python]
import subprocess

def test_user_input_output():
    user_inputs=["Abdulla",22]
    # Run the main program and capture the output
    process = subprocess.Popen(['python', '1.py'], stdin=subprocess.PIPE, stdout=subprocess.PIPE, stderr=subprocess.PIPE, text=True)
    out, err = process.communicate(input='Abdulla\n22\n')  # Simulate user input for name and age
    i=0
    for item in out:
        print(item,end="")
        if(item == ':'):
            print(user_inputs[i])
            i+=1
            print("\n")

if __name__ == "__main__":
    test_user_input_output()
\end{lstlisting}

\begin{enumerate}
    \item \textbf{Importing subprocess:}
    \begin{itemize}
        \item The subprocess module is imported to run external processes (in this case, running another Python script).
    \end{itemize}
    
    \item \textbf{Test Function \texttt{test\_user\_input\_output}:}
    \begin{itemize}
        \item This function simulates user input and captures the output of another Python script (1.py).
    \end{itemize}
    
    \item \textbf{User Inputs Simulation:}
    \begin{itemize}
        \item User inputs are defined as "Abdulla" (name) and 22 (age) to simulate user interaction with the program.
    \end{itemize}
    
    \item \textbf{Running the Main Program:}
    \begin{itemize}
        \item The subprocess.Popen function is used to run the main program (1.py) and capture its output.
    \end{itemize}
    
    \item \textbf{Communicating User Input:}
    \begin{itemize}
        \item The communicate method is used to send user input ("Abdulla\n22\n") to the program being tested.
    \end{itemize}
    
    \item \textbf{Processing Output:}
    \begin{itemize}
        \item The output received from the program is processed character by character in a loop.
        \item If a colon (:) is encountered in the output, it is assumed to be a prompt for user input, and the corresponding user input is printed.
    \end{itemize}
    
    \item \textbf{Test Execution:}
    \begin{itemize}
        \item The \texttt{test\_user\_input\_output} function is executed if the script is run directly (\texttt{\_\_name\_\_ == "\_\_main\_\_"}).
    \end{itemize}
\end{enumerate}

\end{document}
